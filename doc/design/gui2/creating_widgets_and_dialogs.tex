\chapter{Creating new widgets and dialogues}

This chapter dives into the area of creating your own widgets or dialogues. For
this example we use a real widget and a dialogue under development. The widget
being developed is the progress bar and the screen the initial load screen.

First we make the widget, then the screen in order to test the widget. It
happens often that, when you need to add a new widget, you need it for a new
dialogue - so you need to create both for testing. Another option is to add the
widget as dummy item to another existing dialogue. When adding a dialogue for
which all widgets already exist this doesn't matter and you can directly dive
into generating the dialogue.

\section{Creating the widget}
\label{sec:creating_the_widget}

The widget is normally split in 7 different files. Three .cpp/.hpp files and one
WML file. Usually I start to create the new files by simply copy pasting the old
files and pick a rather small class like timage or something. Then, I start to
update the build system files, followed by creating the code for the
classes\footnote{Obviously I already thought about the design of the class
before that, but that's of less importance in this tutorial.}.

Now, we start to explain what the various files do. First, I do a general overview of
the files after which I dive more into the implementation of each file.

\begin{description}

\item[src/gui/auxiliary/widget\_definition/progress\_bar.*]
	These files contain the definition of the progress bar. This mainly involves
	the fixed fields for the widget. This widget has no extra fields so the copy
	pasted version suffices. This file contains the \emph{static} properties for
	a certain definition of a progress bar.

\begin{description}
\item[hpp] \Liref{widget_definition.hpp} contains the sample code.
	\begin{description}
	\item[\Lref{widget_definition.hpp:control}] Normally a
		widget definition inherits from tcontrol\_definition which defines the
		basic mandatory fields for a widget definition and a templated load
		function for the resolution. Most widgets don't add more members to this
		class.
	\item[\Lref{widget_definition.hpp:resolution}] Normally a resolution
		definition inherits from tresolution\_definition\_ which define the
		basic mandatory fields for a resolution definition. Not all widgets use
		these members, but most do. Most widgets don't add more members to this
		class, except most container classes that add a grid builder
		definition. Another example is are the scrollbars which define some
		minimum sizes for parts of their control.
	\end{description}

\item[cpp] \Liref{widget_definition.cpp} contains the sample code.
	\begin{description}
	\item[\Lref{widget_definition.cpp:textdomain}] Every file in the library
		is inside the ``wesnoth-lib'' text domain.
	\item[\Lref{widget_definition.cpp:constructor}] The constructor used is
		pretty typical for all widgets.
	\item[\Lref{widget_definition.cpp:resolution_constructor}] The
		constructor used is semi-typical. Resolutions that have their own
		members initialize them in the constructor, but also validate mandatory
		fields for their existence.

		Also the progress bar only has one state, most widgets have more and
		thus add more states.

		And maybe the most important part of this constructor is the wiki
		comment. This comment is used to generate wiki pages about the widget.
		These wiki pages are used by content creators to know how to create a
		proper WML file for the class.
	\end{description}
\end{description}

\item[data/gui/default/widget/progress\_bar\_default.cfg]
	Now that the widget is defined, we can decide how it should look. Which is
	done in this file. Note that in the formula we use ``percentage'' which is
	the percentage of the bar to be filled.

	\begin{description}
	\item[\Lref{progress_bar.cfg:textdomain}] Every file in the library
		is inside the ``wesnoth-lib'' text domain.

	\item[\Lref{progress_bar.cfg:class}]
		Beginning of the progress bar. This part is much better documented in
		the wiki\footnote{\url{https://wiki.wesnoth.org/GUIWidgetDefinitionWML}}.
		Since this part is aimed at WML authors, duplicating that effort here
		makes no sense so just visit the wiki.

	\end{description}

\item[src/gui/auxiliary/window\_builder/progress\_bar.*]
	These files contain the code to build a widget from the definition described
	above and a config object as defined in the window. These files contain also the
	\emph{dynamic} properties of the progress bar. Since the progress bar has no
	dynamic content this file is also rather short.

\begin{description}
\item[hpp] \Liref{window_builder.hpp} contains the sample code.
	\begin{description}
	\item[\Lref{window_builder.hpp:control}] Normally a
		window builder inherits from tbuilder\_control which defines the
		basic mandatory fields for a window builder.

		The function also declares the build function.

		Widget define their own member if they want to have some runtime
		settings, e.g. the spacer defines its fixed size if needed.
	\end{description}

\item[cpp] \Liref{window_builder.cpp} contains the sample code.
	\begin{description}
	\item[\Lref{window_builder.cpp:textdomain}] Every file in the library
		is inside the ``wesnoth-lib'' text domain.
	\item[\Lref{window_builder.cpp:constructor}] The constructor used is
		pretty typical for all widgets.

		If the constructor initializes custom members it may also add some validation.
		This is also true when the widget has a grid, like the listbox or
		multi\_page.
	\item[\Lref{window_builder.cpp:build}]
		This build() is semi-typical, it creates the widget, initializes the
		default fields and returns the created object.

		Widgets with their own members initialize them after initializing the
		default members.
	\item[\Lref{window_builder.cpp:wiki}]
		At the end of the file it contains two wiki comment sections:

		The first defines a macro with a short description of the widget.

		The second describes the extra fields for the instance of the widget.
	\end{description}
\end{description}

\item[src/gui/widgets/progress\_bar.*]
	These files contain the interaction the widget has with the outer world and
	how it reacts to events. Again, the progress bar is rather boring. The
	interesting part is the setter of the percentage. The value of the
	percentage used in the drawing routines is set here\footnote{Other classes have a
	separate update canvas, but that's because the canvas in those classes is
	invalidated in several functions.}.

\begin{description}
\item[hpp] \Liref{progress_bar.hpp} contains the sample code.
	\begin{description}
	\item[\Lref{progress_bar.hpp:class}]
		Most simple widgets derive from tcontrol, which is the base class of the
		visual widgets. Another common base is tcontainer or
		tscrollbar\_container.

	\item[\Lref{progress_bar.hpp:constructor}]
		The class calls the control constructor with the number of states, which
		is the number of canvases used. Other then that, it does the
		initialization of its members. Some constructors do a bit more
		processing, but where a lot needs to be done it's often deferred to a
		finalize() function.

	\item[\Lref{progress_bar.hpp:inherited}]
		This section implements or overrides member functions of the base class.
		Some functions are purely virtual and need to be overridden by others to
		change the behaviour.\footnote{Of course, adding here a list of functions which are
		common would be nice. However, the library is still too
		volatile to do so - no need to create a soon out of date list.}

	\item[\Lref{progress_bar.hpp:settersgetters}]
		This section contains the setters and getters for the class. The
		list below shows the common convention where T is the type of the
		setter/getter and ``foo'' the name of the member without trailing
		underscore.

		\begin{description}
		\item{setter}
			Signature void set\_foo(const T foo); The type T can be by reference
			if a larger class.

		\item{getter}
			Signature T get\_foo() const { return foo\_; } The type T is always
			returned by value.

		\item{reference}
			Signature T\& foo() { return foo\_; } This version returns the value
			by reference, and it should also have a const version. These
			functions or at least the non-const version are often not public.

		\item{pointer}
			Signature T* foo() { return foo\_; } Almost the same as the
			reference version and is used for members that are pointers. The
			text of the reference also applies here.
		\end{description}

		The setter is normally available. The getter is often omitted when a
		reference or pointer function are used. Especially it there is a const
		version of these functions.

	\item[\Lref{progress_bar.hpp:state}]
		The state has the list of states available and normally has COUNT as
		last value, which in turn is used in the constructor to initialize the
		base class.

	\end{description}

\item[cpp] \Liref{progress_bar.cpp} contains the sample code.
	\begin{description}
	\item[\Lref{progress_bar.cpp:textdomain}] Every file in the library
		is inside the ``wesnoth-lib'' text domain.

	\item[\Lref{progress_bar.cpp:logheader}]
		The common headers used for logging, this makes sure the output of the
		items always have the same format. Used by
		LOG\_FOO \textless\textless{} LOG\_HEADER \textless\textless{} "foo.\textbackslash{}n"; or
		log\_scope(foo, LOG\_SCOPE); % check function signature

	\item[\Lref{progress_bar.cpp:register}]
		Registers a widget, this part is really volatile and still has some
		hoops and rings to jump through. Normally, it works. If not, too bad -
		try to look through the other widgets to see how to fix it.

		\textbf{Caveat emptor} Just when all compiles fine and you suddenly get a runtime
		error about your new nice class, the builder claims your widget doesn't
		exist. \emph{Don't panic}, probably you did nothing wrong. Grep TRY in
		src/gui/auxiliary/window\_builder.cpp (at the moment of writing line 96)
		and read the comment above, duly add your class to the list and be
		happy.

	\item[\Lref{progress_bar.cpp:get_control_type}]
		A simple helper to get the human readable name of the class. The first
		version was inlined in the header, but that seems to be an ODR
		violation. At least at some point a compiler or linker complained, can't
		really remember which one. % TODO git pick-axe

	\end{description}
\end{description}

This completes the writing of the files for the new classes, some files need to
be modified to get things up and running. The build system files need the three
new .cpp files listed, these are:
\begin{itemize}
\item src/Makefile.am
\item src/SConstruct
\item src/CmakeLists.txt
\end{itemize}

The next step has become optional now, when most of build systems can glob the files
needed for translation, but I still add them manually as well. The file is
po/wesnoth-lib/POTFILES.IN. It holds the files for the wesnoth-lib text
domain.

The last file to be edited is src/gui/widgets/settings.cpp. Here, a wiki comment
for the new class needs to be added, this might change later. The reason is that
before automatically registering widgets this file needed more modification, so
it was done in one fell swoop.

\end{description}

This concludes the simple progress bar widget. Of course, we haven't tested it
yet, since there's no dialogue to test it in. So let's implement a dialogue.


\section{Creating the window}
\label{sec:creating_the_window}

The window is normally split in 3 files. One .cpp/.hpp file and one WML file.
The dialogue used is the unit attack dialogue, at least an initial version. This
version barely does what it needs to do, and the in game version will be more
polished. This simple version is nice as an example here since it's not too large.

Now, let's explain what the various files do. First, a general overview of
the files after which I dive more into the implementation of the file.

\begin{description}

\item[src/gui/dialogs/unit\_attack.*pp]
	These files contain the code of the dialogue. This code `binds' the WML
	code and allows the C++ code to show the dialogue show the right data at the
	right place. In several cases this also allows the designer of the WML file
	to decide which widget s/he picks.

\begin{description}
\item[hpp] \Liref{unit_attack.hpp} contains the sample code.
	\begin{description}
	\item[\Lref{unit_attack.hpp:class}]
		Defines the class itself, which normally inherits from tdialog.
		Sometimes a class inherits from another class, but this other class then
derives from tdialog; for example the wml\_dialogs have a common base and
separate classes for left and right.

	\item[\Lref{unit_attack.hpp:constructor}]
		Dialogues are often created and then shown once before being destroyed,
therefore the constructor often takes all the parameters needed. Since the show()
function shouldn't be overloaded, the only other alternative would be member
functions to set the members. Note that when keeping `references' to widgets these
members must be pointers and set to NULL in the constructor, this due to the
simple fact that the widgets aren't yet created at this point. Also keep in mind that
when the dialogue is shown twice (using the \emph{same} object) the widgets won't be
the same. The window owning the widgets is destroyed after being shown and
a new one created upon the next call to show().

	\item[\Lref{unit_attack.hpp:settersgetters}]
		As said before, there are often no setters, since all is set in the
contructor. Most members also remain hidden, since the caller has the
information. Only members that change are `exported'. The most common example in
a class with a listbox, the last selected item is made public, since this is
often used as the choice the user made.

	\item[\Lref{unit_attack.hpp:window_id}]
		This function is a part of the window registration and only needs to be
declared in the header, the definition will be provided by a macro.

	\item[\Lref{unit_attack.hpp:pre_show}]
		This function is called after the window is created, but before being
shown. This is the point to set the members that point to a widget. As said
before, every time the dialogue is shown a newly created window is shown as well, so
setting the pointers can be unconditional. This is also the place to fill the
widgets with the proper content, if needed. E.g. filling the languages in the
language selection dialogue.

	\item[\Lref{unit_attack.hpp:post_show}]
		Called after the dialogue is shown, this allows you to update certain
statuses. E.g. if the dialogue is closed with the OK button set the selected
language to the newly selected language. You can also update it unconditionally
and let the caller test the status.

	\item[\Lref{unit_attack.hpp:members}]
		At this point the list of private members start.

	\end{description}

\item[cpp] \Liref{unit_attack.cpp} contains the sample code. Note some
parts are already described more verbosely in the header and thus not mentioned
here.
	\begin{description}
	\item[\Lref{unit_attack.cpp:textdomain}]
		Like written for the widget, every file is inside the `wesnoth-lib' text
domain.

	\item[\Lref{unit_attack.cpp:wiki}]
		Directly after the includes (inside the gui2 namespace) the wiki
documentation starts.

	\item[\Lref{unit_attack.cpp:register}]
		After the wiki documentation the registration code follows, it defines
the window\_id function and does some other registration parts.

	\item[\Lref{unit_attack.cpp:pre_show}]
		The pre\_show is already mentioned in the header file and it the one
used here uses several static helper functions to do it's job.

	\end{description}

\end{description}

\item[src/gui/dialogs/unit\_attack.cfg]
	This file contains the WML that defines the widget. It should honour the
restrains set in the wiki comment like \lref{unit_attack.cpp:wiki}. These
constrains set the minimum set of needed widgets and their types, and mentions
the optional widgets. How the widgets are placed and whether only the optional
ones are set is up to the designer of the dialog. The gui toolkit wiki has more
information on the subject. Still some parts are worth mentioning.

	\begin{description}
	\item[\Lref{unit_attack.cfg:textdomain}]
		Every file in the library is inside the ``wesnoth-lib'' text domain.

	\item[\Lref{unit_attack.cfg:defines}]
		With larger dialogs, I often first design them on paper and then divide the
dialog in several sections. These sections often are turned into defines in
order to keep the main dialog rather simple. WML tends to be verbose and deeply
nested, and the GUI's also tend to have a lot of nested items, these powers
combined result in deeply nested structure, where one can easily lose the way.
Also note that all local macros are prefixed with \_GUI, this is to avoid clashes
with other macros. The BIG in the names are just because I want to make a
different definition of high resolution screens (this will be the first screen
using that feature, but that will be added later).

	\item[\Lref{unit_attack.cfg:window}]
		Not much to tell about this part, since it's all documented in the wiki. %TODO add ref?
But do note that due to the defines above, the grid in the window itself is
rather simple.

	\item[\Lref{unit_attack.cfg:undefines}]
		The local macros are no longer needed so undefine them, I normally
undefine them in the oposite order of the definitions. I'm quite sure that's not
actually needed, but it feels the right way\texttrademark.

	\end{description}

\end{description}
